    \begin{abstract}

\noindent Los elementos conocidos en la naturaleza muestran variaciones en su estructura nuclear. De especial interés son los radioisótopos; estos son átomos de un elemento que difieren del resto de su especie por el número de neutrones en su núcleo o por el nivel energético del mismo y que se desintegran hasta obtener una estructura estable. El uranio-235 es un radioisótopo natural. Su desintegración se suscita mediante una reacción en cadena en la que cada producto de fisión es inestable y se descompone en otro elemento en su debido turno. El modelo matemático que describe la reacción en cadena del U-235, también llamada serie del actínido, es las ecuaciones de Bateman. Para que la reacción en cadena tome lugar se requiere una masa mínima de material fisionable conocida como masa crítica. En cada proceso de desintegración se libera una cantidad de energía que puede transformarse en electricidad. Utilizamos GNU Octave para simular la serie del actínido mediante los métodos de Euler-Maruyama, Runge-Kutta y Monte Carlo. 
  
    \end{abstract}
    \keywords{Desintegración en cadena, serie del actínido, ecuaciones de Bateman, simulación numérica}