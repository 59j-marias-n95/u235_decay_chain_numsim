Las reacciones nucleares —aquellas en las que participan los núcleos atómicos—, explica \cite{Murray.2020}, pueden tener lugar espontáneamente o inducirse. Estas reacciones son mucho más energéticas que las químicas, pero obedecen a las mismas leyes físicas, continúa \cite{Murray.2020}. 

La literatura clasifica las reacciones nucleares en dos grupos principales: reacciones de fisión y reacciones de fusión, como se puede revisar en \cite{Basdevant.2005, Cottingham.2001, Krane.1987, Murray.2020} entre otros.

\subsection{Reacciones de Fisión}
La fisión nuclear es la ruptura de un núcleo espontáneamente o inducida, de acuerdo con \cite{Basdevant.2005}, en la cual un núcleo pesado se convierte en un sistema de fragmentos más livianos \cite{Murray.2020}. Tal ruptura implica una liberación de energía hacia el entorno, razón por la cual se identifica esta reacción como exotérmica \cite{Basdevant.2005, Murray.2020}. 

Las sustancias radiactivas requieren de un volumen menor de materia que una fuente de combustibles fósiles para generar una cantidad equivalente de energía, de acuerdo con \cite{Sanctis.2016}. La mínima cantidad de materia requerida para sustentar una reacción en cadena para una muestra radiactiva cuyos productos se mantienen inestables se llama \textit{masa crítica}, y su estimación es un problema central en física nuclear aplicada como lo muestra \cite{MoralesBolio.1974}. 

\subsection{Reacciones de Fusión}
La fusión nuclear, en palabras de \cite{Murray.2020} es la combinación de dos o más núcleos atómicos en uno más pesado. Estas reacciones también se manifiestan como exotérmicas, en particular, la energía solar que recibe nuestro planeta es esencialmente producida por este fenómeno, explica \cite{Basdevant.2005}. 

Hasta ahora, un dispositivo práctico de energía de fusión no ha sido demostrado, y considerable investigación y desarrollo será necesario para alcanzar ese objetivo. \cite{Murray.2020}

\subsection{Energía de desintegración}
Siguiendo la notación empleada en \cite{Krane.1987}, una reacción nuclear general se puede expresar como:
\begin{equation*}
  a + X \longrightarrow Y + b  
\end{equation*}
$X$ e $Y$ representan núcleos atómicos, mientras que $a$ y $b$ son partículas que interceden en la reacción ya sea induciendo el proceso o siendo generadas por el proceso. La ley de la conservación de la energía dicta, en el régimen relativista, la siguiente correspondencia entre el antes y el después del proceso nuclear:

\begin{equation*}
    m_Xc^2+T_x+m_ac^2+T_a\ = m_Yc^2+T_Y+m_bc^2+T_b,
\end{equation*}

\noindent donde T representa la energía cinética de las partículas. Las emes, $m_X,\ m_a,\ m_Y,\ m_b$, representan las masas de las partículas en cuestión. Se define el valor Q (energía de la reacción o energía de desintegración) como:
\begin{equation*}
    Q=(m_{inicial}-m_{final})c^2,  
\end{equation*}
\noindent o como el exceso de la energía final de los productos.
\begin{equation*}
   Q=T_{final}-T_{inicial}
\end{equation*}    
