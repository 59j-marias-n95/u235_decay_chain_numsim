    \begin{abstract}

\noindent Los elementos conocidos en la naturaleza muestran variaciones en su estructura nuclear. De especial interés son los radioisótopos; estos son átomos de un elemento que difieren del resto de su especie por el número de nucleones en su núcleo o por el nivel energético del mismo y que debido a la interacción nuclear débil se desintegran hasta obtener una estructura energéticamente más estable. Las aplicaciones de los radioisótopos son diversas, nos concierne su capacidad para generar energía eléctrica. El uranio-235 es un radiosótopo natural. Su desintegración se suscita mediante una reacción en cadena en la que cada producto de fisión es inestable y se descompone en otro elemento en su debido turno. El modelo matemático que describe la reacción en cadena del U-235, también llamada serie del actínido, son las ecuaciones de Bateman. Para que la reacción en cadena tome lugar se requiere una masa mínima de material fisionable conocida como masa crítica. En cada proceso de desintegración se libera una cantidad de energía que puede transformarse en electricidad. Utilizamos GNU Octave para simular la serie del actínido mediante los métodos de Euler, Runge-Kutta y Monte Carlo y los comparamos. Así mismo, determinamos la energía de desintegración en cada paso del proceso. 
  
    \end{abstract}
    \keywords{Desintegración en cadena, cálculo de masa crítica, energía de desintegración, serie del actínido, ecuaciones de Bateman, simulación numérica}