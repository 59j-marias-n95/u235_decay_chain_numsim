Como parte del proceso de desintegración radiactiva, los productos de fisión del uranio emiten neutrones y rayos gamma al atravesar una serie de decaimientos beta y alfa, según \cite{Guo.2016}. La causa de esta serie de desintegraciones nucleares se debe a la razón de neutrones sobre protones, explica \cite{Guo.2016}. 

Es sabido que los núcleos pesados muestran una tendencia a la inestabilidad cuando el número de neutrones supera al de protones \cite{Podgorsak.2016, Krane.1987}.  

De acuerdo con \cite{Guo.2016}, al momento de analizar una cadena de desintegración compleja se simplifica el problema al resolver la reacción a través de reacciones lineales en las que cada núcleo está relacionado solo a un núcleo padre. No obstante, es apreciable de inmediato que la naturaleza estocástica del proceso es obviada con esta suposición.

El proceso de desintegración del uranio-235 contempla bifurcaciones en varios puntos, esto es: hay productos de desintegración que pueden a su vez producir uno de dos posibles núcleos más estables como se documenta en \cite{HUBENER2003211, International_Atomic_Energy_Agency2013-bq}. 

En la literatura científica se registran más bifurcaciones o menos para el proceso, en virtud de los valores de las constantes de ramificación, como podemos apreciar en \cite{HUBENER2003211,International_Atomic_Energy_Agency2013-bq,Pratiwi.2021,Loch.2013}. 

Para mantener la simulación lo más fiel posible al proceso real, se ha tomado como referencia la cadena presentada en \cite{HUBENER2003211}, la cual está ilustrada en la imagen \ref{cadenadelu235} replicada para fines de continuidad de la lectura en anexos.

\subsection{Ecuaciones de Bateman para la serie del actínido}
Tomando en cuenta las condiciones iniciales \ref{condicioninicialpadre} y \ref{condicioninicialproductos} y la forma de $N_i(t)$ en \ref{iesimonucleo}, las ecuaciones de Bateman para la serie del actínido son:
\begin{align}
    N_1'(t)&=-\lambda_1 N_1(t)\\ \label{ecubateman1} %Uranio
    N_2'(t)&=\lambda_1 N_1(t) -\lambda_2 N_2(t)\\ %Torio
    N_3'(t)&=\lambda_2 N_2(t) -\lambda_3 N_3(t)\\ %Paladio
    N_4'(t)&=\lambda_3 N_3(t) -\lambda_4 N_4(t)\\ %Actinio
    N_5'(t)&=(1-f_4)\lambda_4 N_4(t) -\lambda_5 N_5(t)\\ %Torio
    N_6'(t)&=f_4 \lambda_4 N_4(t) -\lambda_6 N_6(t)\\ %Francio
    N_7'(t)&=\lambda_5 N_5(t) + \lambda_6 N_6(t) -\lambda_7 N_7(t)\\ %Radio
    N_8'(t)&=\lambda_7 N_7(t) -\lambda_8 N_8(t)\\ %Radón
    N_9'(t)&=\lambda_8 N_8(t) -\lambda_9 N_9(t)\\ %Polonio
    N_{10}'(t)&=f_9\lambda_9 N_9(t) -\lambda_{10} N_{10}(t)\\ %Plomo
    N_{11}'(t)&=(1-f_9)\lambda_9 N_9(t) -\lambda_{10} N_{11}(t)\\%Astato
   N_{12}'(t)&=\lambda_{10} N_{10}(t) +\lambda_{11} N_{11}(t)\\
   &-\lambda_{12} N_{12}(t)\\ %Bismuto
   %\end{align}
   %
   %\begin{align}
    N_{13}'(t)&=(1-f_{12})\lambda_{12} N_{12}(t)\\
    & -\lambda_{13} N_{13}(t)\\ %Polonio
    N_{14}'(t)&=f_{12}\lambda_{12}N_{11}(t) -\lambda_{14} N_{14}(t)\\ %Talio
    N_{15}'(t)&=\lambda_{13} N_{13}(t) + \lambda_{14} N_{14}(t) %Plomo
    \label{ecubateman15}
\end{align}

\noindent donde $f_k$ son tasas de ramificación. Las $N_i(t)$ se definen de la siguiente manera:
\begin{itemize}
    \item $N_1(t)\equiv$ \textit{Núcleos de} $^{235}U$.
    \item $N_2(t)\equiv$ \textit{Núcleos de} $^{231}Th$.
    \item $N_3(t)\equiv$ \textit{Núcleos de} $^{231}Pa$.
    \item $N_4(t)\equiv$ \textit{Núcleos de} $^{227}Ac$.
    \item $N_5(t)\equiv$ \textit{Núcleos de} $^{227}Th$.
    \item $N_6(t)\equiv$ \textit{Núcleos de} $^{223}Fr$.
    \item $N_7(t)\equiv$ \textit{Núcleos de} $^{223}Ra$.
    \item $N_8(t)\equiv$ \textit{Núcleos de} $^{219}Rn$.
    \item $N_9(t)\equiv$ \textit{Núcleos de} $^{215}Po$.
    \item $N_{10}(t)\equiv$ \textit{Núcleos de} $^{211}Pb$
    \item $N_{11}(t)\equiv$ \textit{Núcleos de} $^{215}At$.
    \item $N_{12}(t)\equiv$ \textit{Núcleos de} $^{211}Bi$.
    \item $N_{13}(t)\equiv$ \textit{Núcleos de} $^{211}Po$.
    \item $N_{14}(t)\equiv$ \textit{Núcleos de} $^{207}Tl$.
    \item $N_{15}(t)\equiv$ \textit{Núcleos de} $^{207}Pb$.
\end{itemize}

\noindent para los factores de ramificación tenemos:
\begin{itemize}
    \item $f_4=\frac{\lambda_{4,\alpha}}{\lambda_{4,\alpha}+\lambda_{4,\beta}}$, donde $\lambda_{4,\alpha}$ es la tasa de desintegración del $^{227}Ac$ hacia $^{223}Fr$, mientras que $\lambda_{4,\beta}$ representa la tasa de desintegración de $^{227}Ac$ hacia $^{227}Th$. 
    \item $f_9=\frac{\lambda_{9,\alpha}}{\lambda_{9,\alpha}+\lambda_{9,\beta}}$, donde $\lambda_{9,\alpha}$ es la tasa de desintegración del $^{215}Po$ hacia $^{211}Pb$, mientras que $\lambda_{9,\beta}$ representa la tasa de desintegración de $^{215}Po$ hacia $^{215}At$.
    \item $f_{12}=\frac{\lambda_{12,\alpha}}{\lambda_{12,\alpha}+\lambda_{12,\beta}}$, donde $\lambda_{12,\alpha}$ es la tasa de desintegración del $^{211} Bi$ hacia $^{207} Tl$, mientras que $\lambda_{12,\beta}$ representa la tasa de desintegración de $^{211} Bi$ hacia $^{211} Po$.
\end{itemize}

De acuerdo con los archivos de \textit{National Nuclear Data Center} en la base de datos \textit{NuDat}, las probabilidades para el canal alfa y el beta en el Ac-227 son 0.013800 y 0.98620, respectivamente; para el canal alfa y beta en el Po-215 son, respectivamente, 0.9999977 y 0.0000023; para el canal alfa y el beta en el Bi-211 son 0.99724 y 0.00276, respectivamente.

Entonces $f_4=0.01380$, $f_9=0.9999977$ y $f_{11}=0.99724$.

La constante de decaimiento $\lambda_i$ para el i-ésimo núcleo depende del período de semi-desintegración $\tau_{1/2}^{(i)}$, estos valores son conocidos y están documentados en la literatura, tal es el caso de \cite{Flanagan1954} y la base datos \textit{NuDat}. 

La tabla \ref{tabladeconstantesdedesintegracion} muestra las constantes de decaimiento de cada núcleo en las ecuaciones \ref{ecubateman1} hasta \ref{ecubateman15} junto con el período de semi-desintegración que le concierne según \textit{NuDat}.

Lo que no se muestra en la tabla \ref{tabladeconstantesdedesintegracion} es las constantes de decaimiento correspondientes a los canales específicos $\alpha$ y $\beta$ en la desintegración del $^227 Ac$, $^215 Po$, y $^211 Bi$. De la definición de razón de fraccionamiento, se puede deducir que

$$\lambda_{4,\alpha}=0.4394\times 10^{-3}\unit{a^{-1}}$$
$$\lambda_{4,\beta}=3.140\times 10^{-3}\unit{a^{-1}}$$
$$\lambda_{9,\alpha}=1.227\times 10^{10}\unit{a^{-1}}$$
$$\lambda_{9,\beta}=2.822\times 10^{4}\unit{a^{-1}}$$
$$\lambda_{11,\alpha}=2.180\times 10^{11}\unit{a^{-1}}$$
$$\lambda_{11,\beta}=6.033\times 10^{8}\unit{a^{-1}}$$

\noindent son las constantes de decaimiento que caracterizan los canales de desintegración observados en la serie del actínido. 

Hay que señalar que en este modelo, las ecuaciones de Bateman tienen una forma determinista. Bajo la perspectiva de un modelo estocástico diferencial, vemos que la solución del sistema sería la solución esperada para el caso determinista más un término perturbativo que es en sí una variable aleatoria dependiente del tiempo. 

En la literatura se reportan métodos analíticos para resolver sistemas de ecuaciones como el planteado aquí empleando transformadas de Laplace o exponenciales de matrices. En este ensayo, se emplea el método de eigenvectores para establecer la solución general del sistema. Al notar que el sistema de ecuaciones puede representarse con una ecuación matricial de la forma

\begin{equation}
	\mathbf{N'}(t)=\Lambda \mathbf{N}(t)\label{sistemadinamicoU235}
\end{equation}


\noindent donde $\mathbf{N}(t)$ es un vector cuya iésima componente es la función $N_i(t)$ del iésimo núcleo en la cadena. El vector derivada $\mathbf{N'}(t)$ es el vector que contiene las derivadas de las funciones iésima. La matriz $\Lambda$ de dimensión $15\times15$ se da en la ecuación \ref{matriz_determinista} en los anexos.

La matriz $\Lambda$ cumple con ser una matriz triangular (inferior), y en ese sentido sus eigenvalores son los elementos de la diagonal principal, es inmediato verificar que tales eigenvalores son distintos y reales, bajo estas condiciones \cite{Hirsch.2004} sostiene que la solución de un sistema de la forma de \ref{sistemadinamicoU235} viene dada como combinación lineal de los eigenvectores de $\Lambda$:

$$\mathbf{N}(t)=\sum_{i=1}^{15}\kappa_i e^{-\lambda_i t} \mathbf{V}_{\lambda_i}$$

\noindent Las constantes de integración $\kappa_i$ son determinadas por las condiciones iniciales $\mathbf{N}(0)=\left[N_1^{(0)}\ 0\ 0\ ...\ 0\right]^T$.

\subsubsection{Cálculo de los coeficientes de Bateman}

De acuerdo con la ecuación \ref{batemangeneral}, podemos emplear las constantes de decaimiento en el cuadro \ref{tabladeconstantesdedesintegracion} para determinar los coeficientes de Bateman de la serie del actínido. No obstante, estas soluciones corresponden a los casos donde la serie de desintegración no presenta ramificaciones, la existencia de tales bifurcaciones en el proceso agrega términos sobre la ecuación \ref{batemangeneral} que no están contemplados en esta forma general.

Una solución simbólica de las ecuaciones por medio de Laplace y Laplace Inverso demuestra que las constantes de decaimiento correspondientes a cada ramificación se combinan de forma alternada en las soluciones deterministas de acuerdo con la aparición de estas bifurcaciones en la cadena.

En el método de eigenvectores primero se obtiene una serie de coeficientes de linealización que acompañan a los vectores linealmente independientes que conforman la solución del sistema, estas constantes se resumen en la tabla \ref{tabladecoeficientes2}. 

Multiplicando tales coeficientes con  los vectores solución (eigenvectores) se recuperan los coeficientes de Bateman para esta serie. Se resumen los resultados en los cuadros \ref{tabla_coeficientes_bateman1} al \ref{tabla_coeficientes_bateman4}. Donde $C_k^*=C_k/N_1^{(0)}$ para simplicidad. 

\subsection{Modelo estocástico matricial}
Un modelo estocástico matricial del sistema se puede construir usando una matriz determinista alternativa $\mathcal{L}$ más tres matrices $\mathcal{X}$, $\mathcal{Y}$ y $\mathcal{Z}$ que conjuntamente representan un ruido estocástico sobre $\mathcal{L}$. El sistema matricial es

\begin{equation}
	\mathcal{N}'(t)=(\mathcal{L}+x\mathcal{X}+y\mathcal{Y}+z\mathcal{Z})\mathcal{N}(t)
\end{equation}

\noindent donde los parámetros $x, y,\textrm{ y }z$ son variables aleatorias discretas que pueden ser 0 o 1, no necesariamente todas a las vez, y dictan la ocurrencia de una bifurcación en la cadena. La forma explícita de estos cuatro arreglos puede verse en los anexos en las ecuaciones \ref{matriz_determinista_alternativa}, \ref{matriz_estocastica_x}, \ref{matriz_estocastica_y} y \ref{matriz_estocastica_z}. El vector $\mathcal{N}$ corresponde al vector de funciones del número de núcleos por unidad de tiempo para cada especie química en la cadena bajo el esquema estocástico. 