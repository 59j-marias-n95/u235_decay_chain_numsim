El método de eigenvectores basado en la discusión de \cite{Hirsch.2004} se implementa mediante la siguiente secuencia de instrucciones en GNU Octave

\begin{lstlisting}
IN: Matriz de coeficientes del 
    sistema dinamico, M.
    Vector de condiciones
    iniciales, N0.

[V, l] = eig(M);
C = linsolve(V, N0);

for i = 1:15
   S(:, i) = C(i)*V(:,i);
endfor
\end{lstlisting}

\noindent donde el arreglo $V$ corresponde a los eigenvectores de $M$ y $l$ a sus eigenvalores. El arreglo $C$ recoge las constantes de integración que satisfacen las condiciones iniciales en ecuaciones \ref{condicioninicialpadre} y \ref{condicioninicialproductos}. Los vectores solución del sistema se almacenan en las columnas del arreglo $S$. 

\subsection{Método de Runge-Kutta Estocástico}

\subsection{Método de Euler-Maruyama}

\subsection{Monte Carlo}